\documentclass[12pt,a4paper]{report}
\usepackage[utf8]{inputenc}
\usepackage[T1]{fontenc}
\usepackage[french]{babel}
\usepackage{materialDesign}
%\ProvidesPackage{materialDesign}
\usepackage{lipsum}
\usepackage[sfdefault]{roboto}  %% Option 'sfdefault' only 
\usepackage{color}
\usepackage{xcolor}
\usepackage{tikz}
\usepackage[explicit]{titlesec}
\usepackage{enumitem}
\usepackage{afterpage}
\usepackage{caption}
\usepackage{graphicx}
\usepackage{tocvsec2}
\usepackage{fancyhdr}
\usepackage{fullpage,lmodern}
\usepackage[normalem]{ulem} % Charger le package ulem pour avoir des soulignements
\usepackage{hyperref}
\usepackage{float}
\usepackage{minted}
\usepackage[tmargin=2cm,headsep=2cm,footskip=15pt]{geometry}
\usepackage{materialDesign}

\usetikzlibrary{shadows.blur}
\usetikzlibrary{shapes.symbols}

\usePrimaryTeal
\useAccentTeal
%\useDarkTheme
\useLightTheme

% Definition des couleurs
\definecolor{accent_green}{HTML}{00E676}
\definecolor{fancystart}{HTML}{93F9B9}
\definecolor{fancyend}{HTML}{38ef7d}
% Couleurs personnalisées listes
\definecolor{azure1}{RGB}{67,233,123}
\definecolor{azure2}{RGB}{56,249,215}
% Couleurs personnalisées background
\definecolor{backgroundstart}{HTML}{232526}
\definecolor{backgroundend}{HTML}{414345}

% Texte par défaut en blanc dans le document
\color{white}

% Personnalisation des cadres (Table des matières)
\hypersetup{
    pdftitle = {Brelot Julien - Internship Report},
    %hidelinks,
    colorlinks=true,
    %linkcolor={red!20!black},
    citecolor=accent_green,
    urlcolor=accent_green,
    linkbordercolor={0 0 0},  % Supprime les bordures autour des liens
    pdfborderstyle={/S/U/W 1} % Ajoute un soulignement aux liens (U = Underline, W = Width of underline)
    % Hide links in the table of contents
}



% Personnalisation des sections
%\titleformat{\section}[block]
%    {\normalfont\Large\bfseries\color{accent_green}}
%    {\thesection}
%    {1em}
%    {}
%\titleformat{\subsection}[block]
%    {\normalfont\large\bfseries\color{accent_green}}
%    {\thesubsection}
%    {1em}
%    {\normalfont\large\bfseries\color{accent_green}{#1}}

% Numérotation des sections en chiffres romains
\renewcommand {\thesection}{\Roman{section}}


\newcommand*\sectiontitle{}
\let\origsection\section
\renewcommand*{\section}[2][]{%
\ifx\setminus#1\setminus% optional argument not present?
  \origsection{#2}%
  \renewcommand*\sectiontitle{#2}%
\else
  \origsection[#1]{#2}%
  \renewcommand*\sectiontitle{#1}%
\fi
}

%Formatte les titres
\titleformat{\section}[hang]
{\Huge \bfseries\sffamily}%
{
    \rlap{
        \color{accent}\rule[-5pt]{\textwidth}{1.2pt}
        }
    \colorbox{accent}{%
            \raisebox{0pt}[18pt][3pt]{ 
                \makebox[40pt]{% height, width
                \fontfamily{phv}\selectfont\color{white}{\thesection}}
            }
    }
}%
{10pt}%
{ \color{accent} \LARGE #1
%
}

% Formatte les sous titres
\titleformat{\subsection}[hang]
{\Large \bfseries\sffamily}%
{
    \rlap{
        \color{accent_green}\rule[-10pt]{0.93\textwidth}{1.2pt}
        }
    \colorbox{accent_green}{%
            \raisebox{0pt}[18pt][3pt]{ 
                \makebox[20pt]{% height, width
                \fontfamily{phv}\selectfont\color{white}{\thesubsection}}
            }
    }
}%
{5pt}%
{ \color{accent_green} \bfseries \large #1
%
}

\titlespacing*{\section}{-30pt}{3mm}{5mm}
%\titlespacing*{\subsection}{0pt}{3mm}{5mm}
\newcommand{\sectionbreak}{\clearpage}
%\renewcommand {\thesubsection}{\arabic{subsection}.}

% Configuration du style de la table des matières
\fancypagestyle{tocstyle}{
    \fancyhf{} % Clear header and footer
    \renewcommand{\headrulewidth}{0pt} % Remove header line
    % Personnalisez ici le style de la page de la table des matières
    % Par exemple, vous pourriez ne rien ajouter ou appliquer un style plus simple.
    %\fancyfoot[C]{\thepage} % Simple page number at the center of the footer
}

% Configuration du style d'en-tête de page
\fancypagestyle{plain}{
    \fancyhf{} % Tout effacer
    \color{black}
    \renewcommand{\headrulewidth}{0pt}
    \fancyhead[C]{
        \begin{tikzpicture}[remember picture,overlay]
            \node[yshift=-1.5cm] at (current page.north west){
                \begin{tikzpicture}[remember picture, overlay]
                    % Rectangle vert en arrière-plan sans le dégradé
                    \draw[fill=accent,blur shadow={shadow blur steps=10}] (0,0) rectangle (\paperwidth,2cm);
                    % Dégradé dans le rectangle avec coins arrondis
                    \node[anchor=center,xshift=.5\paperwidth,yshift=0.8cm, rectangle,rounded corners,inner sep=9pt,left color=fancystart, right color=fancyend, blur shadow={shadow blur steps=10}]
                        {\color{black} \Large \sectiontitle }; % \thesection -
                \end{tikzpicture}
            };
        \end{tikzpicture}
    }

    % Configuration du style de pied de page (numéro de page)
    %\pagestyle{fancy}
    %\fancyhf{} % Clear all header and footer fields
    %\renewcommand{\headrulewidth}{0pt} % Remove header line
    \fancyfoot{%
    \begin{tikzpicture}[remember picture, overlay]
        \fill[left color=azure1,right color=azure2] (current page.south) ++(0em,2em) circle (1.5em);
        \node[anchor=south,text=textAccent, yshift=1.3em] at (current page.south) { \hyperlink{toc}{\bf \thepage}};
    \end{tikzpicture}%
    }
}

\setlength{\headheight}{15pt} % Avoids fancyhdr warning

% Personnalisation des listes
\setlist[enumerate,1]{label=\textcolor{accent}{\textbullet}}
\setlist[itemize,1]{label=\textcolor{accent}{\Large $\bullet$}}



\begin{document}

% Ajouter une cible pour le lien de retour à la table des matières
\addtocontents{toc}{\protect\hypertarget{toc}{}}

% Page de garde
%\pagecolor{backgroundstart}
\begin{titlepage}

    % Dégradé en fond 
    \begin{tikzpicture}[remember picture,overlay]
        \shade[top color=backgroundstart,bottom color=backgroundend] (current page.south west) rectangle ++(\paperwidth,\paperheight);
    \end{tikzpicture}

    % Rectangle en arrière-plan (vert en haut)
    \begin{tikzpicture}[remember picture,overlay]
        \fill[fill=accent] (current page.north west) rectangle ++(\paperwidth,-2cm);
    \end{tikzpicture}
    
    % Rectangle en arrière-plan (vert en bas)
    \begin{tikzpicture}[remember picture,overlay]
        \fill[fill=accent] (current page.south west) rectangle ++(\paperwidth,2cm);
    \end{tikzpicture}
    
    \vfill % Adds vertical space, pushing content down to the center

    % Titre du rapport
    \begin{center}
        {\Huge \bfserie Réseaux Programmables} \\
        \vspace{0.5cm} 
        {\LARGE \bfserie Rapport de projet \par}\\
        \vspace{0.5cm} 
        {\Large Perfectly Optimized Kit for Efficient Monitoring (POKEMON)\par}
    \end{center}
    
    \vspace{1cm} % Adds a little space between title and author section
    
    % Auteur et affiliation
    \begin{center}
        {\LARGE Brelot Julien\par}
        {\Large \textit{ \href{URL}{Git} \par}}
    \end{center}

    \vspace{1.5cm} % Adds space between author info and logos

    % Logos
    \begin{center}
        \begin{minipage}{0.45\textwidth}
            \includegraphics[width=\textwidth]{images/logo-tps.png}
        \end{minipage}
        \hfill
        \begin{minipage}{0.45\textwidth}
            \includegraphics[width=\textwidth]{images/logo_imt.png}
        \end{minipage}
    \end{center}
    
    \vspace{1.5cm} % Adds space between logos and VTEC logo

    
    \vfill % Adds vertical space, pushing content up to the center
    
\end{titlepage}
%\setcounter{section}{31} 

\thispagestyle{tocstyle}



\tableofcontents % Table des matières
\clearpage % Passer à la page suivante

\pagestyle{plain} % Style de page pour le contenu


\section{Introduction}

    \begin{card}
    Ce rapport présente le travail réalisé dans le cadre du mini-projet de Réseaux Programmables. \\
    L'objectif de ce projet est de mettre en place un réseau d'equipements réseaux programmables, avec une solution de supervision et de contrôle des liens. \\
    Dans ce but, le data plane est programmé en P4 avec un fichier source unique et générale pour l'ensemble des équipements. \\
    Le control plane est programmé en Python et permet de contrôler les équipements et de collecter les informations de supervision à l'aide d'un méta-contrôleur. \\
    Les routeurs disposent de 3 types controle plane différent : simple, avec perte et  une dernier définissant un routage non optimal. \\
    \end{card}

\section{Réalisations}

    \subsection{Contenu}

    \begin{card}
        On retrouve dans ce mini-projet : 
        \begin{itemize}
            \item Un rfichier P4 pour le data plane des 3 types de routeurs.
            \item Un méta-contrôleur qui permet de contrôler les routeurs et de collecter les informations de supervision.
            \item 3 types de controle plane définis en python pour les routeurs : simple_router, simple_router_loss et simple_router_stupid.
            \item Un fichier python network.py qui permet de générer et lancer un réseau.
        \end{itemize}
    \end{card}

    \subsection{Fonctionnement}
        \subsubsection{Data Plane}



\section{Conclusion}

    \begin{card}
        Cette étude met en évidence les avantages et inconvénients des architectures NoSQL et relationnelles dans divers scénarios. \\
        MongoDB offre une grande flexibilité et scalabilité, totefois, la réplication et le sharding peuvent impacter les performances des opérations CRUD. \\
        Néanmoins, le sharding, s'il est possible est une très bonne solution. \\
        Avec une meilleure configuration, on pourrait s'attendre à des performances similaire au mode standalone et on a tout de même obtenu des meilleures performances que dans le mode répliqué. \\
        Toutefois MySQL, reste performant pour des cas nécessitant des transactions complexes et ne peut montrer l'étendue de ses performances dans un cas aussi simple ne nécessitant pas de modèle relationnel. \\
        De plus, les performances du cluster MySQL s'est avéré, contre toute attente, plus performant que MySQL Standalone, et aussi performante que MongoDB Standalone. \\
    \end{card}

% On crée une partie appendice pour les annexes ou il y a des informations complémentaires, sur le déploiement du code

\appendix
\section{Annexe}
\subsection{Déploiement du Code}


Le projet peut être cloné depuis le dépôt git suivant : 
\begin{card}
    \begin{minted}{bash}
    git clone https://github.com/BJCode-git/Projet-TDLE.git -b main &&
    cd Projet-TDLE
\end{minted}
\end{card}



\emph{Installation des dépendances python :}

\begin{card}
\begin{minted}{bash}
    pip3 install -r requirements.txt
\end{minted}
\end{card}

\emph{Démarrage du réseau :}

\begin{card}
    \begin{minted}{bash}
    python3 network.py
\end{minted}
\end{card}


\emph{Démarrage du méta-contrôleur et des \emph{simple router} :}

\begin{card}
\begin{minted}{bash}
    python3 controllers/meta_controller.py
\end{minted}
\end{card}


\end{document}